 \noindent The field of \acrfull{er} has grown in importance as \acrfull{ai} and \acrfull{ml} have gained popularity due to its potential to transform a variety of fields and applications \cite{er-appl}. With the ongoing evolution of virtual environments, such as the Metaverse, \acrshort{er} is gaining momentum as a critical component in creating a more immersive and engaging experience. By utilizing AI algorithms to analyze various physiological responses and facial expressions, \acrshort{er} technology has the potential to enhance social interactions and emotional experiences of users in virtual environments. But not only that, \acrshort{er} can offer great benefits in many other fields as well, such as marketing, healthcare, education and \acrfull{hci} \cite{er-appl,healthcare,EGGER201935}. As a contribution to this change, our project aims to develop a system that recognizes emotions based on both facial and speech features. Firstly, we examine what approaches have been taken so far in chapter \ref{chap:related-work}. Then, in chapter \ref{chap:project-ideas}, we define our idea and the resulting goals in more detail. In chapter \ref{chap:methodology}, we address the methodology to achieve those goals. In this context, we will observe our approach for \acrfull{fer} and \acrfull{ser} seperately. Moreover, in the course of the project we will address the the discussed steps and implementation in more detail.
