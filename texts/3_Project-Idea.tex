\noindent 
%The project idea is to use \acrshort{ml} algorithms that can recognize emotions in facial images (FER) and speech (SER). 
The goal of the project is to create a viable, working application that can differentiate between emotions using pictures or possibly a video (FER) and speech (SER) of the user.
The system will be developed using software engineering principles such as modular design, testing, and documentation to ensure reliability, scalability, and maintainability. The system will also incorporate a user feedback mechanism to enable active learning. \\


\noindent 
%As discussed in the previous chapters, there are plenty of areas in which such a solution can be used. It has then become a challenging problem that this field involves more and more scientists with different specializations \cite{HCSI201451}. Thus, the purpose of this project is to create a model that can aid individuals within their work or leisure by giving more context to virtual interactions with the addition of emotions. 
%\\
As discussed in the previous chapters, there are plenty of areas in which such a solution can be used. 
%For our project, we choose virtual education as the context but aim to keep our system as general as possible so that it can be utilized in other scenarios as well. 
%The use of ER in virtual education offers the possibility of dynamically adapting learning units to individual students based on their emotional feedback. 
Specializing the system on one area in particular would require us to gain additional expertise for that area which would extend the scope of the course. Thus, we decided to develop a generalized system that can be adapted to different use cases in the future. 
%By placing the focus on two approaches to ER instead of one, we hope to increase the overall performance of the ER system. 

\noindent In order to distribute the workload, we have decided to divide the group into two subgroups, each taking care of one of the approaches (see Figure \ref{fig:workload_division}). At the beginning, both subgroups will work individually on their approach. Then, to create the working application, the two approaches will be combined by a collaboration between the two subgroups.  
\\



\begin{table}
\begin{center}
\begin{tabular}{ |c|c| } 
 \hline
 Facial & Speech \\
 \hline
 Sarah Ternus & Cyprien Touron Decourteix \\ 
 John Carlsson & Jana Valentina Stefan \\ 
 Lukas Runt & Ergi Kokoshi \\ 
 Philip Le Borne & Simon Dannemann \\
 \hline
\end{tabular}
\caption{Division of workload.}\label{fig:workload_division}
\end{center}
\end{table}



