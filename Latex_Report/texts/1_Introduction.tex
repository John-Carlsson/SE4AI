 \noindent The field of \acrfull{er} has grown in importance as \acrfull{ai} and \acrfull{ml} have gained popularity due to its potential to transform a variety of fields and applications \cite{er-appl}. With the ongoing evolution of virtual environments, such as the Metaverse, \acrshort{er} is gaining momentum as a critical component in creating a more immersive and engaging experience. By utilizing AI algorithms to analyze various physiological responses and facial expressions, \acrshort{er} technology has the potential to enhance social interactions and emotional experiences of users in virtual environments. But not only that, \acrshort{er} can offer great benefits in many other fields as well, such as marketing, healthcare, education and \acrfull{hci} \cite{er-appl,healthcare, EGGER201935}. \par
 \noindent
 As a contribution to this change, our project aims to develop a system that recognizes emotions based on both facial and speech features. Firstly, we examine what approaches have been taken so far in chapter \ref{chap:related-work}. Then, in chapter \ref{chap:project-ideas}, we define our idea and the context of our system. In chapter \ref{chap:problem-spec}, we derive the more detailed problem description as well as the resulting goals and research questions. In chapter \ref{chap:methodology}, we sum up the methodological steps to follow in order to achieve those goals. Chapter \ref{chap:Data} covers the collection of appropriate data and their preprocessing. Then, in chapter \ref{chap:Implementation}, we pursue implementing and training our models. These are then evaluated in \ref{chap:Evaluation}. Once the performance of the models is as desired, the two approaches are combined in the final application in chapter \ref{chap:final-application}. Lastly, the report is completed with a conclusion and an outlook in chapter \ref{chap:conclusion}.

